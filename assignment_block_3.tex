\documentclass[12pt]{article}
\title{MA080G Cryptography Assignment Block 3}
\author{Viktor Rosvall}

\usepackage{amsmath}
\usepackage{mathtools}

\begin{document}
	\maketitle
	
	\section*{Question 3}
	\renewcommand{\theenumi}{\alph{enumi}}
	\renewcommand{\theenumii}{\roman{enumii}}
	\begin{enumerate}
		\setcounter{enumi}{2}
		\item Explain the discrete logarithm problem.
		
		\item Explain the operation of the ElGamal public-key cryptosystem
	\end{enumerate}	


	\section*{Answer 3}
	\renewcommand{\theenumi}{\alph{enumi}}
	\renewcommand{\theenumii}{\roman{enumii}}
	\begin{enumerate}
		\setcounter{enumi}{2}
		\item The discrete logarithm problem is potential solution to the problem of finding the private exponent $d$, such that $x \equiv y^d \text{ (mod \textit{n})}$ in the RSA cryptosystem.
		\\
		\\
		\fbox{
			\parbox{\textwidth}{
				\textbf{Definition:} given $x,y$ and a prime $p$ such that:
				$$
				y \equiv x^e \text{ (mod $p$)}
				$$
				find $e$.
			}
		} 
		\\
		\\
		This problem however is believed to be as hard as factorization and not yet proven to be NP-complete. The order or $x$ should be as large as possible to avoid it being broken by a exhaustive search. So $x$ should be chose as a primitive root mod $p$, which is an element of order $\lambda(p) = p-1$
		
		\item Let prime $p = 23$ and the primitive root $g = 5$.
		
		The ElGamal cryptosystem works as: Bob chooses a prime $p$ and a primitive root $g$ mod $p$. He then chooses a \textit{secret} exponent $a \in \{1,...,p-1\}$. Let $a=3$ and compute 
		$$
		h=g^a \text{ MOD $p$} = 5^3 \text{ MOD 23} = 23
		$$ 
		This gives us the \textit{public key} ($p,g,h$) = ($23,5,10$) and the private component $a = 3$. 
		
		Say Alice want's to send a plaintext message $x$ to Bob, where $x \in \{1,...,p-1\}$. 
		Let $x = 6$. Alice then choose a \textit{secret} exponent $k \in \{1,...,p-1\}$. Let $k = 8$ and compute 
		\[
		\begin{matrix*}[l]
			y_1&=g^k \text{ MOD $p$} & y_2&=xh^k \text{ MOD $p$}\\
			&=5^8 \text{ MOD 23} &&= 6*10^8 \text{ MOD 23}\\
			&=16 &&=12
		\end{matrix*}
		\]
		where $p,g,h$ is Bob's \textit{public key}. This gives Alice the ciphertext pair ($y_1,y_2$) = $(16,12)$.   
		
		Bob receives the message ($y_1,y_2$) = ($g^k,xh^k$) mod $p$. Bob knows his secret number $a$ such that $h=g^a \text{ (mod $p$)}$, so he can thus compute 
		$$
		h^k \equiv (g^a)^k \equiv (g^k)^a \text{ (mod $p$)}
		$$
		Remember that Bob knows $g^k$ from the ciphertext pair Alice sent. He also knows his secret $a$. He can thus easily compute 
		$$
		h^k \equiv (g^k)^a \text{ (mod $p$)} = 16^3 \text{ MOD 23} = 2
		$$
		For Bob to extract $x$ from $xh^k$ he needs to compute the inverse of $h^k$, i.e., $(h^k)^{-1}$ with the EEA of $h^k$ and $p$, which gives us $$
		(h^k)^{-1} = \text{EEA(2,23)} = 12
		$$
		Multiply the inverse of with $xh^k$ 
		$$
		xh^k*(h^k)^{-1} \text{ MOD $p$} = x \text{ MOD $p$} = x
		$$
		$$
		x = 12 * 12 \text{ MOD 23} = 6
		$$
		$$
		x \text{ MOD 23} = 6
		$$
		gives us $x = 6$.
	\end{enumerate}	
	
	
\end{document}