\documentclass{article}
\title{MA080G Cryptography Assignment 1}
\author{Viktor Rosvall}

\usepackage{amsmath}

\begin{document}
	\maketitle
	\subsection*{Question 1}
	Showing you working, decrypt the shift-encrypted message
	PM FVB DHUA AV RLLW H ZLJYLA FVB TBZA HSZV RLLW PA MYVT
	FVBYZLSM.
	
	\subsection*{Answer 1}
	By looking at the first digram PM, I made the guess that the plaintext is "if". So I start the decryption by shifting the alphabet 7 steps to the left for P to become an "f". After that I begin checking if the remaining ciphertext translate to a  plausible English sentence. 
	
	By shifting 7 steps to the left the sentence becomes: "if you want to keep a secret you must also keep it from yourself".
	
	\subsection*{Question 2}
	\subsubsection*{10.6.1}
	Write down the cycle notation for permutation which effect the rearrangement.
	\[
	\begin{matrix}
		1 & 2 & 3 & 4 & 5 & 6 & 7 & 8 & 9 \\
		\downarrow & \downarrow & \downarrow & \downarrow & \downarrow & \downarrow & \downarrow & \downarrow & \downarrow & \\
		3 & 5 & 7 & 8 & 4 & 6 & 1 & 2 & 9
	\end{matrix}
	\]
	
	\subsubsection*{10.6.2}
	Let $\sigma, \tau$ be the permutations of \{1, 2,...,8\} whose effects representations in cycle notation are: 
	$$\sigma = (1 \ 2 \ 3) \ (4 \ 5 \ 6) \ (7 \ 8), \ \ \ \ \tau = (1 \ 3 \ 5 \ 7) \ (2 \ 6) \ (4) \ (8)$$
	Write down the cycle notations for $\sigma\tau,\tau\sigma,\sigma^2,\sigma^{-1},\tau^{-1}$.
	
	\subsubsection*{10.6.4}
	Show that there are just three members of $S_4$ which have two cycles of lenght 2 when written in cycle notation.
	
	\subsubsection*{10.6.5}
	Let $K$ denote the subset of $S_4$ which contains the identity permutation $i$ and the three permutations $\alpha_1,\alpha_2,\alpha_3$ described in the previous exercise. Write out the "multiplication table" for $K$, when multiplication is interpreted as composition of permutations.
	
	\subsection*{Answer 2}
	By using cycle notation to calculate blabal bla
	The inverse of a permutation in cycle notation is the number backwards. One-cycles can be discarded as they will not effect the calculations.
	
	\subsubsection*{10.6.1}
	Cycle notation:	$(1 \ 3 \ 7) \ (2 \ 5 \ 4 \ 8)$
	
	\subsubsection*{10.6.2}
	\[
	\begin{split}
	\sigma\tau & = (1 \ 2 \ 3) \ (4 \ 5 \ 6) \ (7 \ 8) * (1 \ 3 \ 5 \ 7) \ (2 \ 6) \\
	& = (2 \ 4 \ 5 \ 8 \ 7) \ (3 \ 6)
	\end{split}
	\]
	
	\[
	\begin{split}
	\tau\sigma & = (1 \ 3 \ 5 \ 7) \ (2 \ 6) * (1 \ 2 \ 3) \ (4 \ 5 \ 6) \ (7 \ 8) \\
	& = (1 \ 6 \ 4 \ 7 \ 8) \ (2 \ 5)
	\end{split}
	\]
	
	\[
	\begin{split}
	\sigma^2 & = (1 \ 2 \ 3) \ (4 \ 5 \ 6) \ (7 \ 8) * (1 \ 2 \ 3) \ (4 \ 5 \ 6) \ (7 \ 8) \\
	& = (1 \ 3 \ 2) \ (4 \ 6 \ 5)
	\end{split}
	\]
	
	\[
	\begin{split}
	\sigma^{-1} & = (3 \ 2 \ 1) \ (6 \ 5 \ 4) \ (8 \ 7) \\
	& = (1 \ 3 \ 2) \ (4 \ 6 \ 5) \ (7 \ 8)
	\end{split}
	\]
	
	\[
	\begin{split}
	\tau^{-1} & = (7 \ 5 \ 3 \ 1) \ (6 \ 2) \\
	& = (1 \ 7 \ 5 \ 3) \ (2 \ 6)
	\end{split}
	\]
	
	\subsubsection*{10.6.4}
	The number of permutations of type $[2^2]$ in $S_4$ can be calculated as: 
	$$
	S_4 \text{ has }\frac{4!}{2^{2}2!} = 3 \text{ permutations }
	$$
	Where the base is the loop size, and $\alpha$ is the number of loops.
	
	The loops available are:
	\[
	\begin{split}
	\alpha_1 & = (1 \ 2) \ (3 \ 4) \\
	\alpha_2 & = (1 \ 3) \ (2 \ 4) \\
	\alpha_3 & = (1 \ 4) \ (2 \ 3) \\
	i & = (1) \ (3) \ (2) \ (4) 
	\end{split}
	\]
	
	\subsubsection*{10.6.5}
	$K$ is a subset of $S_n$ which contains the permutations $\alpha_1,\alpha_2,\alpha_3, i$ from above. 
	\begin{center}
		\begin{tabular}{ | l | c | r | }
			\hline
			$\odot$ & $i$ & $\alpha_1$ & $\alpha_2$ & $\alpha_3$ \\ \hline
			$i$ & $i$ & $\alpha_1$ & $\alpha_1$\\ \hline
			$\alpha_1$ \\
			$\alpha_2$ & 8 & 9 \\
			$\alpha_3$
			\hline
		\end{tabular}
	\end{center}

	By looking at the multiplication table of K
	
	\subsection*{Question 3}
	In how many ways can you rearrange the letters of the string
	ABRAABRAKADABRA?
	
	\subsection*{Answer 3}
	We have 15 letters in total, but they aren't unique. I count 7 As, 3Bs, 3Rs, 1 K and 1 D.
	If each letter was unique, the number of permutations would be $15!$, but since the number of permutations aren't, they can be counted as:
	$$
	\frac{15!}{7!3!3!} = 7207200
	$$
	
	\subsection*{Question 4}
	\subsubsection*{a) (i)}
	Use Euclid's algorithm to show that $gcd(17,2018)=1$
	\subsubsection*{a) (ii)}
	Find two integers $s$ and $t$ such that:
	$$
	17s+2018t=1
	$$
	
	\subsubsection*{b)}
	Showing all your working, find all solutions $[x] \in Z_{2018}$ to the equation
	$$
	[17] \odot [x] = [1]
	$$
	
	\subsubsection*{c)}
	Showing all your working, find the set of all integers $x$ which satisfy the congruence
	$$
	17x \equiv 1 (mod \ 2018)
	$$
	
	\subsection*{Answer 4}
	\subsubsection*{a) (i)}
	By using Euclid's algorithm step by step we get;
	\begin{align*}
	2018 = 17(118) + 12 & \rightarrow & 12 = 2018-17(118) \\
	17=12(1)+5 & \rightarrow & 5=17-12(1) \\
	12=5(2)+2 & \rightarrow & 2=12-5(2) \\
	5=2(2)+\textbf{1} & \rightarrow & 1=5-2(2) \\
	2=1(2) + 0
	\end{align*}
	So $gcd(17,2018)=1$.

	\subsubsection*{a) (ii)}
	Working backwards through the algorithm step by step we get:
	\begin{align*}
	gcd(17,2018) =5	& = 5-2(2) 		&= 5-[12-5(2)](2) \\
		 			& = 5(5)-12(2) 	&=[17-12(1)](5)-12(2) \\
			 		& = 17(5)-12(7)	&= 17(5)-[2018-17(118)](7) \\ 
			 		& = 17(831)-2018(7) &
	\end{align*}
	So $s = 831$ and $t=7$.
	
	\subsubsection*{b)}
	To solve $[x] \in Z_{2018}$ we need to find the multiplicative inverse of $[17]$. We know that a multiplicative inverse exists because the $gcd(17,2018)=1$. 
	
	From Euclid's algorithm we get that $s = 831$ and $t=7$. The class $[831]$ is thus the multiplicative inverse of $[17]$ 
	$$
	[17] \odot [831] = [14127] = [1]
	$$
	
	\subsubsection*{c)}
	To find the set of all integers $x$ in the congruence:
	$$
	17x \equiv 1 (mod \ 2018)
	$$
	is the same as finding all $[x] \in Z_{2018}$ satisfying the equation:
	$$
	[17] \odot [x] = [1]
	$$
	
	we can calculated the inverse of $[17]$ to be $[831]$, thus the set of all integers is:
	$$
	x \in \{...,-1187,831,2849,...\} \ \ \text{ or } \ \ x \in \{831+2018t \mid t \in Z\}
	$$
	
	\subsection*{Question 5}
	Using tools to help you do the frequency analysis of cipher text. Decipher the following text and submit the first line of the plaintext together with a description of how you broke the cipher.
	
	\subsection*{Answer 5} 
	The most frequent trigram is RTA, XUO and CUB. I make the assumption that they are the most frequent trigrams: "the", "and", "ing". I focused on trigrams such as CUB and XUO which has letters in common.
	
	By looking at the \textit{digraph frequency} tablet, we can see that R, D, Q and K are frequent digrams (such as "ss" and "oo")  with themselves. I thus assume that they are probably the letters "t", "o", "s", "l".
	
	Through much trial and error, by focusing on trigrams mostly, I was able to find all the letter mappings which looks like this:
	\[
	\begin{matrix}
	A&B&C&D&E&F&G&H&I&J \\ 
	\downarrow&\downarrow&\downarrow&\downarrow&\downarrow&\downarrow&\downarrow&\downarrow&\downarrow&\downarrow \\
	e&g&i&o&c&k&u&x&y&w \\ \\
	K&L&M&N&O&P&Q&R&S&T \\ 
	\downarrow&\downarrow&\downarrow&\downarrow&\downarrow&\downarrow&\downarrow&\downarrow&\downarrow&\downarrow \\
	l&m&f&q&d&p&s&t&b&h \\ \\
	U&V&W&X&Y&Z \\
	\downarrow&\downarrow&\downarrow&\downarrow&\downarrow&\downarrow \\
	n&V&r&a&Y&v 
	\end{matrix}
	\]
	The ciphertext letters H and N was hard to crack because they only occurred once in the text. It was by process of elimination that I was able to determine them.
	
	The ciphertext letters V and Y has no occurrences which means that the plaintext letters "j" and "z" never occurs too. \\
	
	The first line of the plaintext is: 
	$$
	\text{"the sun was shining on the sea, shining with all his might"}
	$$
	
\end{document}