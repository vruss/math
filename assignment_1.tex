\documentclass{article}
\title{MA080G Cryptography Assignment 1}
\author{Viktor Rosvall}
\date{Mars 2019}

\usepackage{amsmath}

\begin{document}
	\maketitle
	\subsection*{Question 1}
	Showing you working, decrypt the shift-encrypted message
	PM FVB DHUA AV RLLW H ZLJYLA FVB TBZA HSZV RLLW PA MYVT
	FVBYZLSM.
	
	\subsection*{Answer 1}
	By looking at the first digram PM, I made the guess that the plaintext is "if". So I start the decryption by shifting the alphabet 7 steps to the left for P to become an "f". After that I begin checking if the remaining ciphertext translate to a  plausible English sentence. 
	
	By shifting 7 steps to the left the sentence becomes: "if you want to keep a secret you must also keep it from yourself".
	
	\subsection*{Question 2}
	\subsubsection*{10.6.1}
	Write down the cycle notation for permutation which effect the rearrangement.
	\[
	\begin{matrix}
		1 & 2 & 3 & 4 & 5 & 6 & 7 & 8 & 9 \\
		\downarrow & \downarrow & \downarrow & \downarrow & \downarrow & \downarrow & \downarrow & \downarrow & \downarrow & \\
		3 & 5 & 7 & 8 & 4 & 6 & 1 & 2 & 9
	\end{matrix}
	\]
	
	\subsubsection*{10.6.2}
	Let $\sigma, \tau$ be the permutations of \{1, 2,...,8\} whose effects representations in cycle notation are: 
	$$\sigma = (1 \ 2 \ 3) \ (4 \ 5 \ 6) \ (7 \ 8), \ \ \ \ \tau = (1 \ 3 \ 5 \ 7) \ (2 \ 6) \ (4) \ (8)$$
	Write down the cycle notations for $\sigma\tau,\tau\sigma,\sigma^2,\sigma^{-1},\tau^{-1}$.
	
	\subsubsection*{10.6.4}
	Show that there are just three members of $S_4$ which have two cycles of lenght 2 when written in cycle notation.
	
	\subsubsection*{10.6.5}
	Let $K$ denote the subset of $S_4$ which contains the identity permutation $i$ and the three permutations $\alpha_1,\alpha_2,\alpha_3$ described in the previous exercise. Write out the "multiplication table" for $K$, when multiplication is interpreted as composition of permutations.
	
	\subsection*{Answer 2}
	By using cycle notation to calculate blabal bla
	The inverse of a permutation in cycle notation is the number backwards. One-cycles can be discarded as they will not effect the calculations.
	
	\subsubsection*{10.6.1}
	Cycle notation:	$(1 \ 3 \ 7) \ (2 \ 5 \ 4 \ 8)$
	
	\subsubsection*{10.6.2}
	\[
	\begin{split}
	\sigma\tau & = (1 \ 2 \ 3) \ (4 \ 5 \ 6) \ (7 \ 8) * (1 \ 3 \ 5 \ 7) \ (2 \ 6) \\
	& = (2 \ 4 \ 5 \ 8 \ 7) \ (3 \ 6)
	\end{split}
	\]
	
	\[
	\begin{split}
	\tau\sigma & = (1 \ 3 \ 5 \ 7) \ (2 \ 6) * (1 \ 2 \ 3) \ (4 \ 5 \ 6) \ (7 \ 8) \\
	& = (1 \ 6 \ 4 \ 7 \ 8) \ (2 \ 5)
	\end{split}
	\]
	
	\[
	\begin{split}
	\sigma^2 & = (1 \ 2 \ 3) \ (4 \ 5 \ 6) \ (7 \ 8) * (1 \ 2 \ 3) \ (4 \ 5 \ 6) \ (7 \ 8) \\
	& = (1 \ 3 \ 2) \ (4 \ 6 \ 5)
	\end{split}
	\]
	
	\[
	\begin{split}
	\sigma^{-1} & = (3 \ 2 \ 1) \ (6 \ 5 \ 4) \ (8 \ 7) \\
	& = (1 \ 3 \ 2) \ (4 \ 6 \ 5) \ (7 \ 8)
	\end{split}
	\]
	
	\[
	\begin{split}
	\tau^{-1} & = (7 \ 5 \ 3 \ 1) \ (6 \ 2) \\
	& = (1 \ 7 \ 5 \ 3) \ (2 \ 6)
	\end{split}
	\]
	
	\subsubsection*{10.6.4}
	\[
	\begin{split}
	S_4 & = \frac{4!}{2^{2}2!} \\
	& = 3
	\end{split}
	\]
	
	\subsubsection*{10.6.5}
	
	\subsection*{Question 3}
	In how many ways can you rearrange the letters of the string
	ABRAABRAKADABRA?
	
	
	
	
	
	\subsection*{Answer 3}
	We have 15 letters in total, but they aren't unique. I count 7 As, 3Bs, 3Rs, 1 K and 1 D.
	If each letter was unique, the number of permutations would be $15!$, but since the number of permutations aren't, they can be counted as:
	$$
	\frac{15!}{7!3!3!} = 7207200
	$$

\end{document}