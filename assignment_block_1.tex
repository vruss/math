\documentclass{article}
\title{MA080G Cryptography Assignment Block 1}
\author{Viktor Rosvall}

\usepackage{amsmath}

\begin{document}
	\maketitle
	
	\subsection*{Question 6}
	\subsubsection*{(a)}
	Explain what Friedman’s Index of Coincidence measures and compute it for the
	ciphertext below, hence explain how you can see that the cipher used to encrypt
	it was not monoalphabetic.
	
	\subsubsection*{Answer (a)}
	
	
	\subsubsection*{(c)}
	Decrypt the first line of the ciphertext below, given that it has been encrypted by
	using a Vigen`ere cipher with the keyword \textbf{fishy}.
	$$
	\text{KQJZR YPWMG XPEBQ YJWJY ZOZAR MILPQ JIKFY GITFG YPAUI HWMSB MINLA FCYOR}
	$$
	
	\subsubsection*{Answer (c)}
	Using the formula we can get the ciphertext $y_i$:
	$$
	y_i = (x_i + k_{i MOD n}) MOD 26
	$$
	In this case, we are looking for the plaintext letters $x_i$.
	\\\\
	The key: \textbf{fishy}, corresponds to the sequence (5,8,18,7,24) as $y_i$. 
	\\\\
	If we want to decrypt the cipher text "KQJZR", which in numeric corresponds to "10,16,9,25,17", we start by using $y_0 = 10$ to decrypt K:
	\[
	\begin{split}
	10 	&= (x_0 + 5_{0 MOD 5}) MOD 26 \\
	10	&= (x_0 + 5_0) MOD 26 \\
	5	&= x_0 MOD 26 \\
	x_0 &= 5
	\end{split}
	\]
	So the first letter "K" $\rightarrow$ "f". Using this formula for the remaining 4 ciphertext letters, we get "KQJZR" $\rightarrow$ "first".
	
	
\end{document}