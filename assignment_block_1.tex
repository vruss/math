\documentclass{article}
\title{MA080G Cryptography Assignment Block 1}
\author{Viktor Rosvall}

\usepackage{amsmath}

\begin{document}
	\maketitle
	
	\subsection*{Question 1}
	\subsubsection*{(a)}
	Explain how a substitution cipher works.
	
	\subsubsection*{Answer (a)}
	A \textbf{Substitution cipher} is done by permuting (scrambling) the letters of the alphabet. For example: replacing the letter "a" $\rightarrow$ "K". The key to a substitution cipher is a table of each permutation.
	
	\subsubsection*{(b)}
	What does it mean for a substitution cipher to be
	\\
	(i) monoalphabetic?\\
	(ii) polyalphabetic?
	
	\subsubsection*{Answer (b)}
	(i)\\
	In a monoalphabetic substitution cipher, the letters are always encrypted the same. 
	
	For example: "a" will always be permuted the same, no matter the position in the plaintext. 
	\\\\
	(ii)\\
	In a polyalphabetic substitution cipher, the latter may be encrypted differently depending on it’s position in the plaintext. 
	
	For example: In a Vigenère cipher with a key length of 3, the letter "a" will be permuted differently depending on if it's in position 0,1 or 2 in the plaintext cycle. Like "a$_0$" $\rightarrow$ "K", "a$_1$" $\rightarrow$ "D" and "a$_2$" $\rightarrow$ "Q" 
		
	\subsubsection*{(c)}
	Give a non-trivial example of a monoalphabetic substitution cipher and use it to
	encrypt the plaintext
	\[
	\begin{split}
	\text{Teach thy necessity to reason thus} \\
		\text{There is no virtue like necessity}
	\end{split}
	\]
	
	\subsubsection*{Answer (c)}
	
	
	\subsubsection*{(d)}
	 Explain how to break a monoalphabetic substitution cipher.
	
	\subsubsection*{Answer (d)}
	
	
	\subsection*{Question 4}
	\subsubsection*{(a)}
	Define Euler’s $\Phi$-function.
	
	\subsubsection*{Answer (a)}
	Euler's $\Phi$ function on the natural numbers $n \geq 2$ given by:
	$$
	\Phi(n) =  \text{\# congruence classes } [a] \in Z_n \text{ such that }gcd(a,n = 1)
	$$
	Counts the number of invertible elements of $Z_n$. 
	
	\subsubsection*{(b)}
	Compute $\Phi(17)$, $\Phi(289)$ and $\Phi(221)$.
	
	\subsubsection*{Answer (b)}
	To compute $\Phi(n)$, we first need to prime factorize $n$, and then use the following formula:

	$$
	\Phi(n) = p_{1}^{a_1-1}(p_1-1)p_{2}^{a_2-1}(p_2-1)...p_{r}^{a_r-1}(p_r-1)
	$$
	Where $n = p_{1}^{a_1}p_{2}^{a_2}...p_{r}^{a_r}$ and $p_1,p_2,...,p_r$ are distinct primes and $a_1,a_2,...,a_r > 0$. 
	\\\\
	17 is already prime so we can calculate $\Phi(17)$ as:
	$$
	\Phi(17) = 17^{1-1}(17-1) = 16
	$$

	289 need to be factorized before we can calculate $\Phi$.
	$$
	289 = 17^2
	$$
	$$
	\Phi(289) = 17^{2-1}(17-1) 
	= 17*16 
	= 272
	$$
	
	And the same thing for 221:
	$$
	221 = 13 + 17
	$$
	$$
	\Phi(221) = 13^{1-1}(13-1) + 17^{1-1}(17-1) = 12*16 = 192
	$$
	
	\subsection*{Question 6}
	\subsubsection*{(a)}
	Explain what Friedman’s Index of Coincidence measures and compute it for the
	ciphertext below, hence explain how you can see that the cipher used to encrypt
	it was not monoalphabetic.
	
	\subsubsection*{Answer (a)}
	The \textit{Index of Coincidence} (IOC) $I$, measures the likelihood of picking 2 identical letters, from a text. 
	\\\\
	We can calculate the IOC $I$, by using the counted letters, given in the assignment. Where the  letters: a,b,c...,z are represented as $n_0,n_1,...,n_{25}$, and $n$ is the total number of letters. 
	\[
	\begin{split}
	I &= \frac{\sum_{0}^{25}n_i(n_i-1)}{n(n-1)} \\
	&=\frac{26(26-1)+20(20-1)+...+25(25-1)}{444*443} \\
	&=\frac{8638}{196692} \\
	&\approx0.44
	\end{split}
	\]
	
	\subsubsection*{(c)}
	Decrypt the first line of the ciphertext below, given that it has been encrypted by
	using a Vigen`ere cipher with the keyword \textbf{fishy}.
	$$
	\text{KQJZR YPWMG XPEBQ YJWJY ZOZAR MILPQ JIKFY GITFG YPAUI HWMSB MINLA FCYOR}
	$$
	
	\subsubsection*{Answer (c)}
	Using the formula we can get the ciphertext $y_i$:
	$$
	y_i = (x_i + k_{i MOD n}) MOD 26
	$$
	In this case, we are looking for the plaintext letters $x_i$.
	\\\\
	The key: \textbf{fishy}, corresponds to the sequence (5,8,18,7,24) as $y_i$. 
	\\\\
	If we want to decrypt the cipher text "KQJZR", which in numeric corresponds to "10,16,9,25,17", we start by using $y_0 = 10$ to decrypt K:
	\[
	\begin{split}
	10 	&= (x_0 + 5_{0 MOD 5}) MOD 26 \\
	10	&= (x_0 + 5_0) MOD 26 \\
	5	&= x_0 MOD 26 \\
	x_0 &= 5
	\end{split}
	\]
	So the first letter "K" $\rightarrow$ "f". Using this formula for the remaining 4 ciphertext letters, we get "KQJZR" $\rightarrow$ "first".
	
	
\end{document}