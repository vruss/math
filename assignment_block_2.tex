\documentclass{article}
\title{MA080G Cryptography Assignment Block 1}
\author{Viktor Rosvall}

\usepackage{amsmath}
\usepackage{mathtools}

\begin{document}
	\maketitle
	
	\section*{Question 2}
	\renewcommand{\theenumi}{\alph{enumi}}
	\renewcommand{\theenumii}{\roman{enumii}}
	\begin{enumerate}
		\item Explain the operation of the RSA public-key cryptosystem.
		
		\item Illustrate your explanation by using the primes $p = 13$ and $q = 17$ and secret decryption key $d = 103$ to
		\begin{enumerate}
			\item  decrypt the ciphertext $z = 2$;
			\item compute the public encryption key e corresponding to $d$;
			\item  encrypt the plaintext $m = 2$
  		\end{enumerate}
  	
  	\item Discuss the security of the RSA public-key cryptosystem.
	\end{enumerate}	

	\subsection*{Answer 2}
	\renewcommand{\theenumi}{\alph{enumi}}
	\renewcommand{\theenumii}{\roman{enumii}}
	\begin{enumerate}
		\item 
		
		\item 
		\begin{enumerate}
			\item  
			\item 
			\item  
		\end{enumerate}
		
		\item 
	\end{enumerate}	



	\section*{Question 3}
	\renewcommand{\theenumi}{\alph{enumi}}
	\begin{enumerate}
		\item Let $p \ge 2$ be a prime. Define what it means for an integer $a$ to be a primitive element modulo $p$.
		
		\item Find a primitive element modulo 23 and prove that it is a primitive element.
	\end{enumerate}	

	\subsection*{Answer 3}
	\renewcommand{\theenumi}{\alph{enumi}}
	\begin{enumerate}
		\item 
		
		\item 
	\end{enumerate}	
		

	
	\section*{Question 4}
	\renewcommand{\theenumi}{\alph{enumi}}
	\begin{enumerate}
		\setcounter{enumi}{2}
		\item Let $a$ and $n$ be positive integers and let $n  \ge 2$. Prove that if gcd($a, n$) = 1 then
		$$
		a^{\Phi(n)} \equiv 1 \text{ (mod $n$)}.
		$$
		
		\item Discuss whether the theorem from part (c) can be used as a primality test.
	\end{enumerate}	

	\subsection*{Answer 4}
	\renewcommand{\theenumi}{\alph{enumi}}
	\begin{enumerate}
		\setcounter{enumi}{2}
		\item 
		
		\item 
	\end{enumerate}	



	\section*{Question 6}
	For positive integers $p \ge 2$, Wilson’s Theorem states that 
	$$
	p \text{ is a prime if and only if } (p - 1)!  \equiv -1 \text{ (mod $p$).}
	$$
	\renewcommand{\theenumi}{\alph{enumi}}
	\begin{enumerate}
		\item Prove Wilson’s Theorem.
		
		\item Discuss whether Wilson’s Theorem is suitable as a primality test for finding	primes to use with RSA.
	\end{enumerate}	

\end{document}