\documentclass{article}
\title{MA080G Cryptography Assignment Block 1}
\author{Viktor Rosvall}

\usepackage{amsmath}
\usepackage{mathtools}

\begin{document}
	\maketitle
	
	\section*{Question 2}
	\renewcommand{\theenumi}{\alph{enumi}}
	\renewcommand{\theenumii}{\roman{enumii}}
	\begin{enumerate}
		\item Explain the operation of the RSA public-key cryptosystem.
		
		\item Illustrate your explanation by using the primes $p = 13$ and $q = 17$ and secret decryption key $d = 103$ to
		\begin{enumerate}
			\item  decrypt the ciphertext $z = 2$;
			\item compute the public encryption key e corresponding to $d$;
			\item  encrypt the plaintext $m = 2$
  		\end{enumerate}
  	
  	\item Discuss the security of the RSA public-key cryptosystem.
	\end{enumerate}	

	\subsection*{Answer 2}
	\renewcommand{\theenumi}{\alph{enumi}}
	\renewcommand{\theenumii}{\roman{enumii}}
	\begin{enumerate}
		\item RSA works by generating a public and private key-pairs from very large primes. The public key can be only be used to decrypted data encrypted using the private key, and vice versa. 
		
		\textbf{Encryption} is done in $Z_n$, where $n$ is the product of two primes, $p$ and $q$.
		\\
		\\
		\fbox{
			\parbox{\textwidth}{
				\textbf{RSA Encryption:} given the public key $(n,e) = k_{pup}$, and the plaintext $x$.
				$$
				y = e_{k_{\text{pup}}}(x) \equiv x^e \text{ (mod \textit{n})}
				$$
				where $x,y \in Z_n$, and $e$ is called the encryption exponent or public exponent.
			}
		}
		\\
		\\
		\\
		\textbf{Decryption} is similarly done in $Z_n$.
		\\
		\\
		\fbox{
			\parbox{\textwidth}{
				\textbf{RSA Decryption:} given the private key $d = k_{priv}$, and the plaintext $x$.
				$$
				x = d_{k_{\text{priv}}}(y) \equiv y^d \text{ (mod \textit{n})}
				$$
				where $x,y \in Z_n$, and $d$ is called the decryption exponent or private exponent.
			}
		} 	
		\\
		\\
		\\
		\textbf{Key Generation} of a public key $(n,e) = k_{pup}$ and a private key $d = k_{priv}$. This means we have to calculate $n,e \text{ and }d$.
		\\
		\\
		\fbox{
		\parbox{\textwidth}{
			\textbf{RSA Key Generation}\\ 
			1. Compute $n = p * q$, where $p$ and $q$ are two large primes.\\
			2. Compute $\Phi(n) = (p-1)(q-1)$\\
			3. Choose a large \textbf{public exponent} $e \in \{1,2,...,\Phi(n)-1 \}$ such that the 
			$$
			\text{gcd(}e,\Phi(n)) = 1. 
			$$ 
			4. Compute the \textbf{private exponent} $d$ such that
			$$
			d*e \equiv 1 \text{ (mod } \Phi(n))
			$$
			thus $d = e^{-1}$.
		}
		}
		\\
		\\
		\\
		When calculating Euclid's algorithm for the gcd we can calculate the linear combination calculated from the Extended Euclid's Algorithm (EEA). This gives us both $e$ and $d$.
		
		\item 
		\begin{enumerate}
			\item We are given that $p = 13$, $q = 17$ and $d = 103$. First we need to calculate what integer ring we are working in, i.e., $n = 13* 17 = 221$.
			
			We are given that the ciphertext $y = z = 2$, so to calculate $x$ we use the formula $x = y^d \equiv \text{ (mod } n)$
			
			$2^{103}$ is a pretty big number so we can't really use calculators on it, especially not on even bigger numbers. So we need to reduce $d$ to something smaller. This can be achieve by using this rule:
			$$
			ab \text{ MOD } n = ((a \text{ MOD } n)(b \text{ MOD } n)) \text{ MOD } n
			$$
			
			So we begin by reducing $2^{103} \text{ MOD } 221$:
			\[
			\begin{split}
				x&=2^{103} \text{ MOD } 221 = ((2^{50} \text{ MOD } 221)(2^{53} \text{ MOD } 221)) \text{ MOD } 221\\
				&=((2^{25} \text{ MOD } 221)(2^{25} \text{ MOD } 221)(2^{25} \text{ MOD } 221)(2^{28} \text{ MOD } 221)) \text{ MOD } 221\\
				&=(2*2*2*16) \text{ MOD } 221\\
				&=2^7 \text{ MOD } 221\\
				&=128
			\end{split}
			\]
			and $x = 128$ is the plaintext as $x = 2^{103} \equiv 2^7 \equiv 128 \equiv \text{ (mod } 221)$
			
			\item Recall from the generation of keys, that $e$ is chosen from the gcd($e, \Phi(n)$) = 1. The EEA of ($e, \Phi(n)$) also gave us $d$. 
			
			We know that $d$ is the inverse of $e$ since $d*e \equiv 1 \text{ (mod } \Phi(n))$. Thus we can calculate $e$ by doing the EEA of ($d, \Phi(n)$) as they are inverses in $Z_n$.
			
			First we begin by calculating the EEA of ($103, \Phi(221)$):
			\[
			\begin{matrix*}[l]
				192&=103 * 1 + 89 & \quad89& = 192 -103(1)\\
				103&=89*1+14 &\quad 14&=103-89(1)\\
				89&=14*6+5 & \quad5 &= 89-14(6)\\
				14&=5*2+4 &\quad 4 &=14-5(2)\\
				5&=4 *2 +1 &\quad 1 &=5-41)
			\end{matrix*}
			\]
			Then we can calculate the linear equation $1 = s*\Phi(n) + t*d$, where $t$ is the inverse of $d$. 
			\[
			\begin{split}
				1&=5-4(1)\\
				&=5(3)-14(1)\\
				&=89(3)-14(19)\\
				&=89(22)-103(19)\\
				&=192(22)-103(41)
			\end{split}
			\] 
			Thus $d^{-1} = e = t = 41$.
			

			\item From (i) and (ii) we have calculated that $n=221$ and $e=43$. The formula for encrypting plaintext $x$ into ciphertext $y$ is:
			$$
			y \equiv x^e \text{ (mod \textit{n})}
			$$ 
			The plaintext $x = m = 2$ encrypted is:
			$$
			y=2^{43} \text{ (mod \textit{n})}
			$$
			
			which can be calculated the same way as we did calculating the decryption
			\[
			\begin{split}
			y&=2^{43} \text{ MOD } 221 = ((2^{20} \text{ MOD } 221)(2^{23} \text{ MOD } 221)) \text{ MOD } 221\\
			&=(152*111) \text{ MOD } 221\\
			&=16872 \text{ MOD } 221\\
			&=76
			\end{split}
			\]
			
		\end{enumerate}
		
		\item Since RSA is based on the concept of multiplying two large primes to get the modulus $n$. If we are able factorize $n$, we could calculate $\Phi(n)$ which is required if we want to find inverse of the public key (as we did in b. (ii)). 
	\end{enumerate}	

	\medspace

	\section*{Question 3}
	\renewcommand{\theenumi}{\alph{enumi}}
	\begin{enumerate}
		\item Let $p \ge 2$ be a prime. Define what it means for an integer $a$ to be a primitive element modulo $p$.
		
		\item Find a primitive element modulo 23 and prove that it is a primitive element.
	\end{enumerate}	

	\subsection*{Answer 3}
	\renewcommand{\theenumi}{\alph{enumi}}
	\begin{enumerate}
		\item 
		
		\item 
	\end{enumerate}	
		
	\medspace
	
	\section*{Question 4}
	\renewcommand{\theenumi}{\alph{enumi}}
	\begin{enumerate}
		\setcounter{enumi}{2}
		\item Let $a$ and $n$ be positive integers and let $n  \ge 2$. Prove that if gcd($a, n$) = 1 then
		$$
		a^{\Phi(n)} \equiv 1 \text{ (mod $n$)}.
		$$
		
		\item Discuss whether the theorem from part (c) can be used as a primality test.
	\end{enumerate}	

	\subsection*{Answer 4}
	\renewcommand{\theenumi}{\alph{enumi}}
	\begin{enumerate}
		\setcounter{enumi}{2}
		\item 
		
		\item 
	\end{enumerate}	

	\medspace

	\section*{Question 6}
	For positive integers $p \ge 2$, Wilson’s Theorem states that 
	$$
	p \text{ is a prime if and only if } (p - 1)!  \equiv -1 \text{ (mod $p$).}
	$$
	\renewcommand{\theenumi}{\alph{enumi}}
	\begin{enumerate}
		\item Prove Wilson’s Theorem.
		
		\item Discuss whether Wilson’s Theorem is suitable as a primality test for finding	primes to use with RSA.
	\end{enumerate}	

	\subsection*{Answer 6}
	\renewcommand{\theenumi}{\alph{enumi}}
	\begin{enumerate}
		\item 
		
		\item 
	\end{enumerate}	

\end{document}