\documentclass{article}
\title{MA080G Cryptography Summary of Block 0 Theory}
\author{Viktor Rosvall}

\usepackage{amsmath}

\begin{document}
	\maketitle
	\subsection*{The main types of encryption}
	In \textbf{Transportation ciphers} encryption is done by changing the ordering of letters in plaintext systematically. A \textbf{Substitution cipher} is done by scrambling the letters of a plaintext. An example of this is the \textit{Caesar cipher}, which encrypts plaintext by shifting the letters of the alphabet 3 times to the right (the key), and decrypts by shifting 3 times to the left. This is called a \textit{Shift cipher} and isn't very secure due to the low key-space. There are 2 kinds of Substitution ciphers: \textit{monoalphabetic} (letters are always encrypted the same) and \textit{polyalphabetic} (a latter may be encrypted differently depending on it’s position in the plaintext). There are 3 kinds of attacks on ciphers: \textit{ciphertext-only}, \textit{known-plaintext} and \textit{chosen-plaintext}. A cipher must be able to withstand a chosen-plaintext attack. 
	
	\subsection*{Permutations}
	Let $ N_{n} = \{1,2,3,...,n\} $ be an alphabet with $n$ letter.
	A permutation of plaintext can be seen as a bijective function: $ \alpha : N_{n} \to N_{n} $ 	

	Permutations can be written in both \textbf{matrix notation}:
	$$
	\alpha = 
	\begin{pmatrix}
	1 & 2 & 3 & 4 & 5 \\
	2 & 1 & 5 & 3 & 4
	\end{pmatrix}
	$$
	and \textbf{cycle notation}, also called \textit{disjoint cycle notation}:
	$$
	\alpha = (1 \ 2) \ (3 \ 5 \ 4)
	$$
	
	The product of two permutations $\alpha, \beta : N_n \to N_n$ is the composite function $\alpha \bullet \beta$, defined as:
	$$
	\alpha \bullet \beta(x) = \alpha(\beta(x)) \ \ \ \forall x \in N_n
	$$
	
	The \textit{inverse} of $\alpha^{-1}$ can be found by swapping the rows in a matrix notation and ordering them. The product of $\alpha$ and $\alpha^{-1}$ is the \textit{identity} permutation $i$ of $N_n$.
	
	\subsubsection*{Counting}
	$S_n$ is the set of all permutations of $N_n$. $S_n$ is called the \textit{symmetric group of degree $n$}. The number of permutations can be counted as $n!$ which is the order of $S_n \ \ \ \forall n \in Z_+$.
	
	The number of permutations of type $[1^{\alpha_1}2^{\alpha_2}...n^{\alpha_n}]$ can be calculated as: 
	$$
	\frac{n!}{1^{\alpha_1}2^{\alpha_2}...n^{\alpha_n}\alpha_1\alpha_2...\alpha_n} 
	$$ 
	Where the base is the loop size, and $\alpha$ is the number of loops
	
	Example: The number of permutations of type $[2^33^2]$ on $S_{12}$ can be calculated as: 
	$$
	\frac{12!}{2^{3}3^{2}3!2!}
	$$
	
	A \textbf{k-cycle} in $S_n$ is a permutation which moves $k$ elements of $N_n$ in a cycle and does nothing to the remaining elements os $N_n$.
	
	\subsubsection*{Relations}
	\textbf{Main Theorem 12.5 in [Biggs]} \\
	Two permutations in $S_n$ are conjugate if and only if they have the same type. \\
	
	The \textit{conjugacy relation $\sim$} is an \textit{equivalence relation} on $S_n$, is defined by $\alpha \sim \beta$ if $\alpha,\beta$ are conjugate permutations in $S_n$.
	
	The \textit{transposition} $T$ is always of type 2: $[2^1]$. It can be used to split cycles or combine them. When you apply a transposition, you always get one more or less cycles. \\
	
	\textbf{Theorem 12.6.2} \\
	Let $n > 2$ be an integer. Half the permutations in $S_n$ are even and half are odd.
	
	\subsection*{Modulo n Arithmetic}
	\textbf{The congruence relation (mod n):}
	$$
	\forall a,b \in Z \ \ \ a \equiv b (mod \ n) \ \  \iff \ \ n|(a-b)
	$$
	
	$\forall x \in Z$ there is a \textit{unique} $r \in \{0,1,...,n-1\}$ such that
	$$
	x \equiv r (mod \ n)
	$$
	
	\textbf{MOD n definition: }\\
	For any positive integer n define a function called MOD n (\% operator):
	$$
	MOD \ n : Z \rightarrow \{0,1,...,n-1\}
	$$
	is given by the rule 
	$$
	x \ MOD \ n = r
	$$
	if $r \in \{0,1,...,n-1\}$ and $x \equiv r (mod \ n)$
	
	\subsubsection*{Euclid's algorithm}
	We can use Euclid's algorithm to find the \textit{greatest common divider}, (gcd) of two integers. The gcd can be calculated by using the Division Theorem on successive remainders, covered in Discrete Mathematics block 3, or by a recursive one-line program [Cameron 2.7]:
	$$
	\textbf{if} \ b = 0 \ \textbf{then} \ gcd(a,b)= a \ \textbf{else} \ gcd(a,b) = gcd(b,a \ MOD \ b) \ \textbf{fi}
	$$
	
	For example:
	$$
	gcd(30,8)=gcd(8,6)=gcd(6,2)=gcd(2,0)=2
	$$
	
	By running the algorithm from block 3 backwards we can find the linear combination of 2 integers. 
	
	\subsubsection*{Euler’s function}
	Only elements $[a] \in Z_n$ with $gcd(a,n) = 1$ has a multiplicative inverse. \\
	
	\textbf{$\varphi$ definition: }\\
	Euler's $\varphi$ function is the function on the natural numbers $n \geq 2$ given by:
	$$
	\varphi(n) =  \text{\# congruence classes } [a] \in Z_n \text{ such that }gcd(a,n = 1)
	$$
	So $\varphi$ counts the number of invertible elements of $Z_n$. \\
	
	\textbf{Theorem} \\
	Let $n = p_{1}^{a_1}p_{2}^{a_2}...p_{r}^{a_r}$ where $p_1,p_2,...,p_r$ are distinct primes and $a_1,a_2,...,a_r > 0$. Then:
	$$
	\varphi(n) = p_{1}^{a_1-1}(p_1-1)p_{2}^{a_2-1}(p_2-1)...p_{r}^{a_r-1}(p_r-1)
	$$
	
	Example:
	$$
	20 = 2^2*5 \text{ \ \ \ so \ \ } \varphi(20) = 2^{2-1}(2-1)5^{1-1}(5-1) = 2*1*4 = 8
	$$
	
	\subsection*{Monoalphabetic Substitution ciphers}
	We permute the plaintext, substituting each letter to a number. In the English alphabet we have a key-space of $26!$. 
	
	This key-space is too large for a brute force attack, but it's still possible to easily break a monoalphabetic cipher by looking at the \textit{letter frequencies} in the ciphertext. By replacing the letters of digrams and trigrams in the ciphertext, to letters that's common in the English alphabet, we can break the cipher. 

~\end{document}