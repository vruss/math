\documentclass{article}
\title{MA080G Cryptography Summary of Block 0 Theory}
\author{Viktor Rosvall}
\date{Mars 2019}

\usepackage{amsmath}

\begin{document}
	\maketitle
	\subsection*{The main types of encryption}

	In \textbf{Transportation ciphers} encryption is done by changing the ordering of letters in plaintext systematically. A \textbf{Substitution cipher} is done by scrambling the letters of a plaintext. An example of this is the \textit{Caesar cipher}, which encrypts plaintext by shifting the letters of the alphabet 3 times to the right (the key), and decrypts by shifting 3 times to the left. This is called a \textit{Shift cipher} and isn't very secure due to the low key-space. There are 2 kinds of Substitution ciphers: \textit{mono-alphabetic} (letters are always encrypted the same) and \textit{poly-alphabetic} (a latter may be encrypted differently depending on it’s position in the plaintext). There are 3 kinds of attacks on ciphers: \textit{ciphertext-only}, \textit{known-plaintext} and \textit{chosen-plaintext}. A cipher must be able to withstand a chosen-plaintext attack. 
	
	\subsection*{Permutations}
	Let $ N_{n} = \{1,2,3,...,n\} $ be an alphabet with $n$ letter.
	A permutation of plaintext can be seen as a bijective function: $ \alpha : N_{n} \to N_{n} $ 
	

	Permutations can be written in both \textbf{matrix notation}:

	$$
	\alpha = 
	\begin{pmatrix}
	1 & 2 & 3 & 4 & 5 \\
	2 & 1 & 5 & 3 & 4
	\end{pmatrix}
	$$
	
	 and \textbf{cycle notation}, also called \textit{disjoint cycle notation}:
	
	$$
	\alpha = (1 \ 2) \ (3 \ 5 \ 4)
	$$
	
	The product of two permutations $\alpha, \beta : N_n \to N_n$ is the composite function $\alpha \bullet \beta$, defined as:
	
	$$
	\alpha \bullet \beta(x) = \alpha(\beta(x)) \ \ \ \forall x \in N_n
	$$
	
	The \textit{inverse} of $\alpha^{-1}$ can be found by swapping the rows in a matrix notation and ordering them. The product of $\alpha$ and $\alpha^{-1}$ is the \textit{identity} permutation $i$ of $N_n$.
	
	$S_n$ is the set of all permutations of $N_n$. $S_n$ is called the \textit{symmetric group of degree $n$}. The number of permutations can be counted as $n!$ which is the order of $S_n \ \ \ \forall n \in Z_+$.
	
	A \textbf{k-cycle} in $S_n$ is a permutation which moves $k$ elements of $N_n$ in a cycle and does nothing to the remaining elements os $N_n$.
	
	$S_n$ has... page 16 part 1. 

\end{document}