\documentclass{article}
\title{MA080G Cryptography Summary of Block 1 Theory}
\author{Viktor Rosvall}

\usepackage{amsmath}

\begin{document}
	\maketitle
	
	\subsection*{Symmetric Ciphers}
	Symmetric Ciphers are divided into \textbf{Stream Ciphers} and \textbf{Block Ciphers}.
	
	\textit{Block Ciphers} encrypt entire blocks of bits at a time with the same key, such as AES-128. Block Ciphers are used more often on the Internet.
	
	\textit{Stream Ciphers} encrypts bits individually. This is done by adding (the same as XOR) a bit from the \textit{key stream} to a plaintext bit (modulus 2). 
	

	\subsection*{The One-Time Pad}
	\textit{The One-Time Pad (OPD)} is a perfect cipher. Meaning, unconditional security cipher: it cannot be broken even with infinite computational resources.
	
	\subsubsection*{Defintion: OTP}
	\textit{A stream cipher for which}
	\begin{enumerate}
		\item the key stream $s_0,s_1,s_2,...$ is generated by a \textit{true} random number generator (TNRG)
		\item the key stream is only known to the legitimate communicating parties
		\item every key stream bit $s_i$ is used only once
	\end{enumerate} 
	
	The need for a TNRG means that the device has to have the capability to produce truly random bits, such as from a dice roll or white noise. 
	
	To ensure that the key stream isn't leaked, we must have a perfectly secure channel of communication.
	
	The fact that each key bit is only used ones means that there is an equal amount of plaintext as keys bits
	
	\subsubsection*{Improving substitution ciphers}
	Substitution ciphers can be solved using frequency analysis, so we need to improve the cipher.
	
	This can be done by inserting additional symbols into the ciphertext with no meaning. Use a different alphabet to represent the ciphertext. Combining transposition with substitution. 
	
	\subsection*{The Vigenère cipher}
	A different \textit{Ceasar Shift} is applied to each letter of the plaintext.
	
	The key to this cipher is a sequence of numbers, explaining the shift length. If the plaintext is longer than the sequence, it repeats. 
	
	Example: The letters "enemy" $\rightarrow$ "JBBQQ" encrypted using the sequence (5,14,23,4,18). 
	
	The trick to remembering the sequence, is to shift the letter "a" to these numbers, for example: "aaaaa"  "FOXES"	
	
	\subsubsection*{Breaking the Vienère cipher}
	If we know the sequence length, we can split the ciphertext into multiple strings of the sequence length, and individually crack them.  
	\\\\
	Suppose we have the plaintext encoded by numbers: $x_0,x_1,x_2,...$. And the key $k_0,k_1,k_2,...,k_{n-1}$ of length $n$  
	
	Using the formula we can get the ciphertext $y_i$:
	$$
	y_i = (x_i + k_{i MOD n}) MOD 26
	$$
	
	For example:
	Assume we have the key: "fishy", which corresponds to the sequence (5,8,18,7,24). 
	
	If we want to decrypt the cipher text "KQJZR", which in numeric corresponds to the "10,16,9,25,17", we would apply the formula 5 times:
	\[
	\begin{split}
		10 	&= (x_0 + 5_{0 MOD 5}) MOD 26 \\
		10	&= (x_0 + 5_0) MOD 26 \\
		5	&= x_0 MOD 26 \\
		x_0 &= 5
	\end{split}
	\]
	So the first letter "K" $\rightarrow$ "f". Continuing this process, "KQJZR" $\rightarrow$ "first".
	
	\subsubsection*{Kasiski method for guessing the key length in a Vienère cipher}
	Using the \textit{Kasiski method} we can guess the key length $n$, by finding the greatest common divisor of the most common digrams and trigrams.
	
	After we have guessed the key length $n$, we can divide the cipher text into $n$ substrings. We can find the key letter by doing a frequency analysis on each column (representing each key letter). 
		
\end{document}