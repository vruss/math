\documentclass{article}
\title{MA080G Cryptography Summary Block 2}
\author{Viktor Rosvall}

\usepackage{amsmath}
\usepackage{mathtools}


\begin{document}
	\maketitle
	
	\section*{Fermat's Little Theorem} 
	Fermat's Little Theorem is useful in primality testing and in public-key cryptography. It can also be used for find the inverse of  an integer $a$ modulo a prime. \cite{fermatsummary}
	\\
	\\
	\fbox{
		\parbox{\textwidth}{
			\textbf{Theorem: }let $a$ be an integer and $p$ be a prime, then:
			$$
			a^p \equiv a \text{ (mod \textit{p})}
			$$
		}
	}
	\\

	This can also be rewritten as: 
	$$
	a^{p-1} \equiv 1 \text{ (mod \textit{p})}
	$$
	
	If $p$ is a prime then the inverse of $a$ can be calculated as:
	$$
	a^{-1} \equiv a^{p-2} \text{ (mod \textit{p})}
	$$
	
 	\subsection*{Proof using modular arithmetic \cite{fermatproof}}
 	asd . 
 	
 	
 	\subsubsection*{The cancellation law}
 	We can cancel out $a$ because p does not divide $a$, nor $k$.
 	
 	\subsubsection*{The rearrangement property}
 	abada

	\newpage

	\begin{thebibliography}{99}
		
		\bibitem{fermatsummary}
		C. Paar, J. Pelzl, 
		\textit{\underline{Understanding Cryptography}}. 2010 ed.
		Springer., Chapter 6.3.4 
		
		\bibitem{fermatproof} 
		Wikipedia, "Proofs of Fermat's little theorem",
		\\\texttt{https://en.wikipedia.org/wiki/Proofs\_of\_Fermat\%27s\_little\_theorem} 18-04-2019
		
	\end{thebibliography}
\end{document}

