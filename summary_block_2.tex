\documentclass{article}
\title{MA080G Cryptography Summary Block 2}
\author{Viktor Rosvall}

\usepackage{amsmath}
\usepackage{mathtools}


\begin{document}
	\maketitle
	
	\section*{Public-key cryptography}
	One of the problems public-key cryptography solves is the \textbf{key distribution problem} by using a distributed \textit{public} key for encryption and a \textit{private} key for decryption.
	
	This works because encryption is done using a \textit{One-way function}.
	\\
	\\
	\fbox{
		\parbox{\textwidth}{
			\textbf{One-way function:} a function $f()$ is a one-way function if:
			\[
			\begin{split}
				\text{1. }& y = f(x) \text{ is computaitonally easy}\\
				\text{2. }&x = f^{-1}(y) \text{ is computaitonally impossible}\\
			\end{split}
			\]
		}
	}
	\\ 
	This means that even if the public key used to encrypt a message is know, it can't be decrypted without the private key. \cite{pubkeysummary}
	
	\subsubsection*{Key Distribution Example \cite{pubkeysummary}}
	Let's say Alice wants to send $x$ to Bob. Both Alice and Bob have a public and private key-pair: $k = (k_{\text{pup}},k_{\text{priv}})$. 
	
	Alice encrypts $x$ using Bob's public key $b_{\text{pup}}$, as:
	$$
	y = e_{b_{\text{pup}}}(x)
	$$ 
	where $e$ is a one-way function. Now Bob can decrypt the received message $y$  using his private key $b_{\text{priv}}$ and retrieve $x$, as:
	$$
	x = d_{b_{\text{priv}}}(y)
	$$
	We can send any data securely using this method. It's common to send key's for symmetric ciphers such as AES, since it's computationally heavy to use these computations. 
	
	\section*{RSA} 
		
	\section*{Man-in-the-middle attack} 
	
	\section*{Fermat's Little Theorem} 
	Fermat's Little Theorem is useful in primality testing and in public-key cryptography. It can also be used for find the inverse of  an integer $a$ modulo a prime. \cite{fermatsummary}
	\\
	\\
	\fbox{
		\parbox{\textwidth}{
			\textbf{Theorem: }let $a$ be an integer and $p$ be a prime, then:
			$$
			a^p \equiv a \text{ (mod \textit{p})}
			$$
		}
	}
	\\

	This can also be rewritten as: 
	$$
	a^{p-1} \equiv 1 \text{ (mod \textit{p})}
	$$
	
	If $p$ is a prime then the inverse of $a$ can be calculated as:
	$$
	a^{-1} \equiv a^{p-2} \text{ (mod \textit{p})}
	$$
	
 	\subsection*{Proof using modular arithmetic \cite{fermatproof}}
 	Let's assume $a$ is a positive integer, not divisible by prime $p$. If we write down the sequence of numbers in modulo $p$
 	$$
 	a,2a,3a,...,(p-1)a
 	$$
 	and after reducing each integer modulo $p$, we get the resulting sequence of numbers
 	$$
 	1,2,3,...,p-1.
 	$$
 	Which means the two sequences are congruent modulo $p$ 
 	$$
 	a,2a,3a,...,(p-1) \equiv 1,2,3,...,p-1 \text{ (mod \textit{p})}
 	$$
 	Which is the same as 
 	$$
 	a^{p-1}(p-1)! \equiv (p-1)! \text{ (mod \textit{p})}.
 	$$
 	After canceling out the sequence of both sides we get
 	$$
	a^{p-1} \equiv 1 \text{ (mod \textit{p})}
 	$$
 	
 	\subsubsection*{Example}
 	Let $a = 2$ and $ p = 7$. The sequence of numbers thus is
 	$$
 	2,4,6,8,10,12
 	$$
 	and after reducing each integer modulo $p$, we get
 	$$
 	2,4,6,1,3,5
 	$$
 	reordered as
 	$$
 	1,2,3,4,5,6.
 	$$
 	The two sequences are also congruent 
 	\[
 	\begin{split}
	 	2,4,6,1,3,5 &\equiv 1,2,3,4,5,6 \text{ (mod \textit{p})}\\
	 	2^6 6! &\equiv 6! \text{ (mod \textit{p})} \\
	 	2^6 &\equiv 1 \text{ (mod \textit{p})}
 	\end{split}
 	\]
 	
 	\subsection*{Euler's generalization \cite{fermatsummary}}
 	Euler's generalization of Fermat's Little Theorem allows any integer modulo $m$, instead of just modulo prime.
 	\\
 	\\`
 	\fbox{
 		\parbox{\textwidth}{
 			\textbf{Euler's Theorem: }let $a$ and $m$ be co-prime integers, i.e., gcd($a,m$) = 1, then:
 			$$
 			a^{\Phi(m)} \equiv 1 \text{ (mod \textit{m})}
 			$$
 		}
 	}
 	
 	\subsubsection*{Example}
 	Let $a = 3$ and $m = 8$. The gcd($3, 8$) = 1. 
 	\\
 	First we need to calculate $\Phi(8)$.
 	$$
 	\Phi(8) = \Phi(2^3) = 2^3-2^2 = 4.
 	$$
 	Now we can use Euler's theorem:
 	$$
 	3^{\Phi(8)} = 3^4 = 81 \equiv 1 \text{ (mod \textit{8})}
 	$$
 	
	\section*{Compute the order of elements in Zp} 
	
	\section*{Carmichael's lambda-function} 
	
	\section*{prove the existence of primitive elements in Zp}
	
	\section*{complexity involved in Primality testing}
	
	\subsection*{Miller-Rabin probabilistic primality test}
	
	\section*{Pollard's p-1 factorisation method}
	 

	\newpage

	\begin{thebibliography}{99}
		
		\bibitem{pubkeysummary}
		C. Paar, J. Pelzl, 
		\textit{\underline{Understanding Cryptography}}. 2010 ed.
		Springer., Chapter 6.1 
		
		\bibitem{fermatsummary}
		C. Paar, J. Pelzl, 
		\textit{\underline{Understanding Cryptography}}. 2010 ed.
		Springer., Chapter 6.3.4 
		
		\bibitem{fermatproof} 
		Wikipedia, "Proofs of Fermat's little theorem",
		\\\texttt{https://en.wikipedia.org/wiki/Proofs\_of\_Fermat\%27s\_little\_theorem} 18-04-2019
		
	\end{thebibliography}
\end{document}

