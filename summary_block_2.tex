\documentclass{article}
\title{MA080G Cryptography Summary Block 2}
\author{Viktor Rosvall}

\usepackage{amsmath}
\usepackage{mathtools}


\begin{document}
	\maketitle
	
	\section*{Public-key cryptography}
	One of the problems public-key cryptography solves is the \textbf{key distribution problem} by using a distributed \textit{public} key for encryption and a \textit{private} key for decryption.
	
	This works because encryption is done using a \textit{One-way function}.
	\\
	\\
	\fbox{
		\parbox{\textwidth}{
			\textbf{One-way function:} a function $f()$ is a one-way function if:
			\[
			\begin{split}
				\text{1. }& y = f(x) \text{ is computaitonally easy}\\
				\text{2. }&x = f^{-1}(y) \text{ is computaitonally impossible}\\
			\end{split}
			\]
		}
	}
	\\ 
	This means that even if the public key used to encrypt a message is know, it can't be decrypted without the private key. \cite{pubkeysummary}
	
	\subsubsection*{Key Distribution Example \cite{pubkeysummary}}
	Let's say Alice wants to send $x$ to Bob. Both Alice and Bob have a public and private key-pair: $k = (k_{\text{pup}},k_{\text{priv}})$. 
	
	Alice encrypts $x$ using Bob's public key $b_{\text{pup}}$, as:
	$$
	y = e_{b_{\text{pup}}}(x)
	$$ 
	where $e$ is a one-way function. Now Bob can decrypt the received message $y$  using his private key $b_{\text{priv}}$ and retrieve $x$, as:
	$$
	x = d_{b_{\text{priv}}}(y)
	$$
	We can send any data securely using this method. It's common to send key's for symmetric ciphers such as AES, since it's computationally heavy to use these computations. 
	
	\subsection*{With Digital Signatures \cite{digitalsignature}}
	A digital signature is a must when using public-key cryptography. Because encryption keys are public, Oscar is able to do a \textbf{man-in-the-middle attack}, i.e., encrypt a message using \textit{Bob's public }key, while faking it's origin as Alice.
	
	This can be fixed using \textbf{digital signatures}. The sender creates a digital signature by using one's \textit{private} key to encrypt (instead of decrypt) a plaintext. Then the sender encrypts the already encrypted plaintext using the receiver's public key, with a message like "Hey, it's bob". 
	
	The receiver decrypts the first layer using his private key and sees the message "Hey, it's bob". At this point the receiver expects the message to be sent from Bob. Thus the receiver will then decrypts the second layer using the sender's (presumably Bob's) public key. 
	
	\section*{RSA} 
	RSA works by generating a public and private key-pairs from very large primes. The public key can be only decrypted using the private key, and vice versa. \cite{rsa-sum}
	
	\subsection*{Encryption and Decryption \cite{rsa-crypt}}
	Encryption is done in $Z_n$, where $n$ is the product of two primes, $p$ and $q$.
	\\
	\\
	\fbox{
		\parbox{\textwidth}{
			\textbf{RSA Encryption:} given the public key $(n,e) = k_{pup}$, and the plaintext $x$.
			$$
			y = e_{k_{\text{pup}}}(x) \equiv x^e \text{ (mod \textit{n})}
			$$
			where $x,y \in Z_n$, and $e$ is called the encryption exponent or public exponent.
		}
	}
	\\
	\\
	\\
	Decryption is similarly done in $Z_n$.
	\\
	\\
	\fbox{
		\parbox{\textwidth}{
			\textbf{RSA Decryption:} given the private key $d = k_{priv}$, and the plaintext $x$.
			$$
			x = d_{k_{\text{priv}}}(y) \equiv y^d \text{ (mod \textit{n})}
			$$
			where $x,y \in Z_n$, and $d$ is called the decryption exponent or private exponent.
		}
	} 	
	
	\subsection*{Key Generation \cite{rsa-generation}}
	We want to generate a public key $(n,e) = k_{pup}$ and a private key $d = k_{priv}$. This means we have to calculate $n,e \text{ and }d$.
	\\
	\\
	\fbox{
		\parbox{\textwidth}{
			\textbf{RSA Key Generation}\\ 
			1. Compute $n = p * q$, where $p$ and $q$ are two large primes.\\
			2. Compute $\Phi(n) = (p-1)(q-1)$\\
			3. Choose a large \textbf{public exponent} $e \in \{1,2,...,\Phi(n)-1 \}$ such that the 
			$$
			\text{gcd(}e,\Phi(n)) = 1. 
			$$ 
			4. Compute the \textbf{private exponent} $d$ such that
			$$
			d*e \equiv 1 \text{ (mod } \Phi(n))
			$$
			thus $d = e^{-1}$.
		}
	}
	\\
	\\
	\\
	When calculating Euclid's algorithm for the gcd we can calculate the linear combination calculated from the Extended Euclid's Algorithm (EEA). This gives us both $e$ and $d$.
	
	\section*{Fermat's Little Theorem} 
	Fermat's Little Theorem is useful in primality testing and in public-key cryptography. It can also be used for find the inverse of  an integer $a$ modulo a prime. \cite{fermatsummary}
	\\
	\\
	\fbox{
		\parbox{\textwidth}{
			\textbf{Theorem: }let $a$ be an integer and $p$ be a prime, then:
			$$
			a^p \equiv a \text{ (mod \textit{p})}
			$$
		}
	}
	\\

	This can also be rewritten as: 
	$$
	a^{p-1} \equiv 1 \text{ (mod \textit{p})}
	$$
	thus $a$ is a \textit{primitive element} mod $p$ \cite{numbertheory}.  \\
	\\
	If $p$ is a prime then the inverse of $a$ can be calculated as:
	$$
	a^{-1} \equiv a^{p-2} \text{ (mod \textit{p})}
	$$
	
 	\subsection*{Proof using modular arithmetic \cite{fermatproof}}
 	Let's assume $a$ is a positive integer, not divisible by prime $p$. If we write down the sequence of numbers in modulo $p$
 	$$
 	a,2a,3a,...,(p-1)a
 	$$
 	and after reducing each integer modulo $p$, we get the resulting sequence of numbers
 	$$
 	1,2,3,...,p-1.
 	$$
 	Which means the two sequences are congruent modulo $p$ 
 	$$
 	a,2a,3a,...,(p-1) \equiv 1,2,3,...,p-1 \text{ (mod \textit{p})}
 	$$
 	Which is the same as 
 	$$
 	a^{p-1}(p-1)! \equiv (p-1)! \text{ (mod \textit{p})}.
 	$$
 	After canceling out the sequence of both sides we get
 	$$
	a^{p-1} \equiv 1 \text{ (mod \textit{p})}
 	$$
 	
 	\subsubsection*{Example}
 	Let $a = 2$ and $ p = 7$. The sequence of numbers thus is
 	$$
 	2,4,6,8,10,12
 	$$
 	and after reducing each integer modulo $p$, we get
 	$$
 	2,4,6,1,3,5
 	$$
 	reordered as
 	$$
 	1,2,3,4,5,6.
 	$$
 	The two sequences are also congruent 
 	\[
 	\begin{split}
	 	2,4,6,1,3,5 &\equiv 1,2,3,4,5,6 \text{ (mod \textit{p})}\\
	 	2^6 6! &\equiv 6! \text{ (mod \textit{p})} \\
	 	2^6 &\equiv 1 \text{ (mod \textit{p})}
 	\end{split}
 	\]
 	
 	\subsection*{Euler's generalization \cite{fermatsummary}}
 	Euler's generalization of Fermat's Little Theorem allows any integer modulo $m$, instead of just modulo prime.
 	\\
 	\\`
 	\fbox{
 		\parbox{\textwidth}{
 			\textbf{Euler's Theorem: }let $a$ and $m$ be co-prime integers, i.e., gcd($a,m$) = 1, then:
 			$$
 			a^{\Phi(m)} \equiv 1 \text{ (mod \textit{m})}
 			$$
 		}
 	}
 	
 	\subsubsection*{Example}
 	Let $a = 3$ and $m = 8$. The gcd($3, 8$) = 1. 
 	\\
 	First we need to calculate $\Phi(8)$.
 	$$
 	\Phi(8) = \Phi(2^3) = 2^3-2^2 = 4.
 	$$
 	Now we can use Euler's theorem:
 	$$
 	3^{\Phi(8)} = 3^4 = 81 \equiv 1 \text{ (mod \textit{8})}
 	$$
 	
	\section*{Compute the order of elements in Zn \cite{groups}} 
	To compute the order or \textit{cardinality} of $Z_n$, the group needs to be finite. $Z_n$ is defined as the set of integers $\{0,1,2,...,n-1\}$. The order of $Z_n$ is denoted as $|Z_n| = n$.
	
	$Z_n^*$ is defined as the set of positive integers, co-prime to $n$. The order of $Z_n^*$ is denoted as $Z_n^* = \Phi(n)$
	\\
	\\
	The order of an element $a$, of a group G, is denoted as: ord($a$). The element $a$ is called the \textit{generator} or a \textit{primitive element} of the group G, if ord($a$) = $|G|$.
	
	\subsubsection*{Example}
	Determine the order of $a = 2$ in $Z_7$. We calculate this by doing modular arithmetic on $2^n$, until $2^n \equiv 1 \text{ (mod \textit{7})}$
	\[
	\begin{matrix*}{}
		2^1&=2 &\equiv 2 \text{ (mod \textit{7})}\\
		2^2&=4 &\equiv 4 \text{ (mod \textit{7})}\\
		2^\textbf{3}&=8 &\equiv 1 \text{ (mod \textit{7})}\\
	\end{matrix*}
	\]
	So from the last line, we can see $2^\textbf{3}=8 \equiv 1 \text{ (mod \textit{7})}$, thus ord(2) = 3 in $Z_7$. This also means that $a = 2$ in $Z_7$ is not a primitive element.
	
	\section*{Carmichael's lambda-function \cite{carmite}} 
	
	
	\section*{Miller-Rabin probabilistic primality test}
	
	\section*{Pollard's p-1 factorization method}
	 

	\newpage

	\begin{thebibliography}{99}
		
		\bibitem{pubkeysummary}
		C. Paar, J. Pelzl, 
		\textit{\underline{Understanding Cryptography}}. 2010 ed.
		Springer., Chapter 6.1 
		
		\bibitem{digitalsignature}
		P. J. Cameron, 
		\textit{\underline{Notes on cryptography}}.
		\\\texttt{http://www.maths.qmul.ac.uk/~pjc/notes/crypt.pdf}
		Chapter 4.4 
		
		\bibitem{rsa-sum}
		C. Paar, J. Pelzl, 
		\textit{\underline{Understanding Cryptography}}. 2010 ed.
		Springer., Chapter 7.1 
		
		\bibitem{rsa-crypt}
		C. Paar, J. Pelzl, 
		\textit{\underline{Understanding Cryptography}}. 2010 ed.
		Springer., Chapter 7.2
		
		\bibitem{rsa-generation}
		C. Paar, J. Pelzl, 
		\textit{\underline{Understanding Cryptography}}. 2010 ed.
		Springer., Chapter 7.3 
		
		\bibitem{fermatsummary}
		C. Paar, J. Pelzl, 
		\textit{\underline{Understanding Cryptography}}. 2010 ed.
		Springer., Chapter 6.3.4 
		
		\bibitem{numbertheory}
		P. J. Cameron, 
		\textit{\underline{Notes on cryptography}}.
		\\\texttt{http://www.maths.qmul.ac.uk/~pjc/notes/crypt.pdf}
		Page 89  
		
		\bibitem{fermatproof} 
		Wikipedia, "Proofs of Fermat's little theorem",
		\\\texttt{https://en.wikipedia.org/wiki/Proofs\_of\_Fermat\%27s\_little\_theorem} 18-04-2019
		
		\bibitem{groups}
		C. Paar, J. Pelzl, 
		\textit{\underline{Understanding Cryptography}}. 2010 ed.
		Springer., Chapter 8.2.2 
		
		\bibitem{carmite}
		P. J. Cameron, 
		\textit{\underline{Notes on cryptography}}.
		\\\texttt{http://www.maths.qmul.ac.uk/~pjc/notes/crypt.pdf}
		Page 91
		
	\end{thebibliography}
\end{document}

