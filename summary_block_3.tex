\documentclass[12pt]{article}
\title{MA080G Cryptography Summary Block 3}
\author{Viktor Rosvall}

\usepackage{amsmath}
\usepackage{mathtools}

\begin{document}
	\maketitle
	
	
	\subsection*{Discrete Logarithm problem}
	
	
	\subsection*{Knapsack problem \cite{knapsack-problem}}
	Let's say we have a \textit{knapsack} with a volume of $b$ units, and a list of items $a_1, a_2,...,a_k$. We want to know if we can fill the knapsack with \textit{some} of the items. 
	
	We want to find a tuple $e$ of length $k$, where $e \in \{0,1\}$, and 
	$$
	\sum_{i = 0}^{k}e_ia_i = b.
	$$
	The knapsack problem is NP since we can easily check if a solution is correct. Finding this solution is hard. We have in the worst-case $2^k$ possible $e$ tuples to check. 
	
	Explain NP-Complete
	
	\subsection*{Merkle-Hellman knapsack cipher}
	
	
	\subsection*{ElGamal cryptosystem}
	
	
	\subsection*{Sophie-Germain primes}
		
	
	\newpage
	\begin{thebibliography}{99}
		
		\bibitem{knapsack-problem}
		P. J. Cameron, 
		\textit{\underline{Notes on cryptography}}.
		\\\texttt{http://www.maths.qmul.ac.uk/~pjc/notes/crypt.pdf}
		Page 78-80  
		
	\end{thebibliography}
\end{document}