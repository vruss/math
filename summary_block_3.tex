\documentclass[12pt]{article}
\title{MA080G Cryptography Summary Block 3}
\author{Viktor Rosvall}

\usepackage{amsmath}
\usepackage{mathtools}
\usepackage{url}

\begin{document}
	\maketitle
	
	
	\subsection*{Discrete Logarithm problem \cite{discrete-log}}
	The discrete logarithm problem is potential solution to the problem of finding the private exponent $d$, such that $x \equiv y^d \text{ (mod \textit{n})}$ in the RSA cryptosystem.
	\\
	\\
	\fbox{
		\parbox{\textwidth}{
			\textbf{Definition:} given $x,y$ and a prime $p$ such that:
			$$
			y \equiv x^e \text{ (mod \textit{})}
			$$
			find $e$.
		}
	} 
	\\
	\\
	This problem however is believed to be as hard as factorization and not yet proven to be NP-complete. The order or $x$ should be as large as possible to avoid it being broken by a exhaustive search. So $x$ should be chose as a primitive root mod $p$, which is an element of order $\lambda(p) = p-1$
	
	
	\subsection*{Knapsack problem \cite{knapsack-problem}}
	Let's say we have a \textit{knapsack} with a volume of $b$ units, and a list of items ($a_1, a_2,...,a_k$). We want to know if we can fill the knapsack with \textit{some} of the items. 
	
	We want to find a tuple $e$ of length $k$, where $e \in \{0,1\}$, and 
	$$
	\sum_{i = 0}^{k}e_ia_i = b
	$$
	where $b$ is the ciphertext.
	
	The knapsack problem is NP since we can easily check if a solution is correct. Finding this solution is hard. We have in the worst-case $2^k$ possible $e$ tuples to check. 
	
	In the case of a \textit{super-increasing} data series $a$, the knapsack-problem degrades into an \textit{easy} problem, so it's not always NP-complete. But it's considered \textit{hard} since we classify problems of it's worst-case behavior. 
	
	A \textbf{super-increasing} sequence is defined as a series of positives integers where each term is greater than the sum of it's predecessors, 
	$$
	\sum_{j=1}^{i-1}a_j<a_i
	$$
	For example $1,2,4,8$ is a super-increasing sequence.
	
	
	
	\subsection*{Merkle-Hellman knapsack cipher \cite{knapsack-cipher}}
	To encrypt using a Merkle-Hellman knapsack cipher we need to create a super-increasing sequence ($a_1,a_2,...,a_k$) which will be our \textit{private key} component. To create a public key component we need to \textit{disguise} the sequence so it can't be broken using the greedy-algorithm. 
	
	To do this we need to choose an integer $n$ greater than the sum of the $a_i$ sequence and an integer $u$ such that gcd($n,u$) = 1, then compute:
	$$
	a_i^* = ua_i \text{ MOD } n
	$$
	for each $a$, creating a new sequence ($a_1^*,a_2^*,...,a_k^*$) which will be our \textit{public key}.
	
	
	\subsection*{ElGamal cryptosystem}
	
	
	
	\subsection*{Sophie-Germain primes \cite{sophie-germain}}
	Using the properties of some special primes, we can easily find a primitive root. A prime number pair ($q,p$) is called a \textit{Sophie-Germain} pair if:
	$$
	p = 2q + 1
	$$
	\fbox{
		\parbox{\textwidth}{
			\textbf{Proposition:} let ($q,p$) be a \textit{Sophie-Germain} pair. Suppose that $1<x<p-2$. Then $x$ is a primitive root mod $p$ if and only if:
			$$
			x^q \equiv -1 \text{ (mod )} p
			$$ 
		}
	}
		
	
	\newpage
	\begin{thebibliography}{99}
		
		\bibitem{discrete-log}
		P. J. Cameron, 
		\textit{\underline{Notes on cryptography}}.
		\\\texttt{http://www.maths.qmul.ac.uk/\textasciitilde{}pjc/notes/crypt.pdf}
		Page 78-80  
		
		\bibitem{knapsack-problem}
		P. J. Cameron, 
		\textit{\underline{Notes on cryptography}}.
		\\\texttt{http://www.maths.qmul.ac.uk/\textasciitilde{}pjc/notes/crypt.pdf}
		Page 78-80  
		
		\bibitem{knapsack-cipher}
		P. J. Cameron, 
		\textit{\underline{Notes on cryptography}}.
		\\\texttt{http://www.maths.qmul.ac.uk/\textasciitilde{}pjc/notes/crypt.pdf}
		Page 80-82
		
		\bibitem{sophie-germain}
		P. J. Cameron, 
		\textit{\underline{Notes on cryptography}}.
		\\\texttt{http://www.maths.qmul.ac.uk/\textasciitilde{}pjc/notes/crypt.pdf}
		Page 108-109
		
	\end{thebibliography}
\end{document}